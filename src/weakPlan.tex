\subsection{Weak Planning Semantics}

The presented semantics is based on a global state operational semantics, that
is, the operational semantics rules and the computation of possible interactions between
timed components is achieved through global states. 
Considering a distributed context, components are intrinsically concurrent and their execution is asynchronous.
This means that even if states of components participating in 
an interaction $\alpha$ are known, $\alpha$ cannot be executed in the global state semantics until the states of all components are known,
which breaks the principle of distribution.
Usually, components are mapped at different
areas on the distributed platform in a way that better suits their interactions. In other terms,
components that synchronize their actions are more likely to be next to each others. However, there are
cases where several components participate in the same interaction but are mapped far from each other,
which adds on communication delays to the interaction corresponding to the exchange of messages. 

In order to reach an efficient scheduling, able of taking decisions ahead and using only partial (local) information,
we define a different semantics based on
a local planning of interactions. It aims to alleviate the problem of communication delays through
an early decision making mechanism while preserving deadlock freedom property of the system. This is achieved by planning each interaction ahead, which means
to choose an execution time within a certain horizon for each interaction, based only on the states of components involved in that interaction.
Consequently, components are notified ahead through communication primitive,
and will wait until the chosen execution time to perform their corresponding actions.
Our approach is to define for each interaction its earliest planning date, which correspond to the maximum horizon value that ensure
a safe planning of the considered interaction.



%In order to localize the most scheduling of interactions and alleviate the problem of communication delays 
%we define a different semantic based on
%a local planning of interactions. It aims to plan each interaction ahead, which implies
%to choose an execution time within a certain horizon for each interaction, based only on the states of components involved in that interaction.
%Consequently, components are notified ahead
%and will wait until the chosen date to execute their corresponding actions.
%Our approach is to define for each interaction its earliest planning date, which correspond to the maximum horizon value that ensure
%a safe planning of the considered interaction.
\subsubsection*{Preliminaries}\label{subsec:wp}\mbox{}\\ 
  We define the predicate $\enabledfrom(\alpha)$ characterizing all states
  from which $\alpha$ is \emph{enabled} if time progresses by $\delta$ units of time, that is:
  \begin{equation}\label{eq:enf}
    \enabledfrom(\alpha)=\bigvee_{\loc\in\Loc_{\alpha}} \big(\at{\loc}\wedge\bigwedge_{a_i\in\alpha}
    (\guard{a_i}{\loc_i}+\delta)\big), 
\end{equation}
%Notice that for $(\loc,\val),(\loc',\val')\in\reach(S)$, and $a_i\in\alpha$ we have:
%$$\enabledfrom(\alpha) \text{ at }(\loc,\val)\Rightarrow \enabled(\alpha) \text{ at } (\loc',\val'),\text{ such that }
%\loc_i=\loc'_i \text{ and } \val_i=\val'_i.$$

\begin{property}\label{pt:enf1}
  Let $(\loc,\val)$ be a state of the composition $S$. For any interaction $\beta\in\gamma$ such that, $\p{\alpha}\cap\p{\beta}=\emptyset$
  and $(\loc,\val)\transit{\beta}_{\gamma}(\loc',\val')$, where $\p{\alpha}$ (resp. $\p{\beta}$) represents components participating in 
  interaction $\alpha$ (resp. $\beta$), if $\enabledfrom(\alpha)$ holds at state $(\loc,\val)$ then it still holds at state $(\loc',\val')$.
\end{property}
This property derives from the fact that executing interactions with disjoint set of components than $\alpha$ does not change the states
of components participating in $\alpha$, that is, for $a_i\in\alpha$ we have $\loc_i=\loc'_i$ and $\val_i=\val'_i$.
\begin{property}\label{pt:enf2}
  Let $(\loc,\val)$ and $(\loc,\val+\delta')$, with $\delta'\in\realpos$ be two states of the composition $S$. 
  If $\enabledfrom(\alpha)$ is $\true$ at state $(\loc,\val)$ then $\enabledfrombis{\delta-\delta'}(\alpha)$ is true at state
  $(\loc,\val+\delta')$ for $\delta'\le\delta$.
\end{property}
This property can be found directly by writing Equation~\ref{eq:enf} on state $(\loc,\val+\delta')$.
%It insists on the fact that a time progress of $\delta'\le\delta$ does not affect the predicate 
%$\enabledfrom(\alpha)$.

Let $\deltamax$ be a partial function $\deltamax: \gamma \to \mathbb{R}_{\ge 0}$ that defines for each
interaction a maximum horizon to be planned with. We define the predicate $\enabledfromd(\alpha)$ characterizing all states from which 
$\alpha$ can be planned with a $\deltama$-horizon as follows:  
\begin{displaymath}
    \enabledfromd(\alpha)=\bigvee_{\loc\in\Loc_{\alpha}} (\at{\loc}\wedge\backwardd(\bigwedge_{a_i\in\alpha}
    \guard{a_i}{\loc_i})), 
\end{displaymath}
  with $\backwardd$ represents an adaptation of the backward operators~\cite{tripakis98:thesis} that satisfies:
\begin{displaymath}
\backwardd g(x)\Leftrightarrow \exists\delta\le\deltama . g (x + \delta),
\end{displaymath}
\begin{property}\label{pt:delta}
  If the predicate $\enabledfrom(\alpha)$ is $\true$ at a state $(\loc,\val)$, then the 
  predicate $\enabledfromd(\alpha)$ is also $\true$ for $\delta\le\deltama$.
\end{property}

\begin{definition}[Plan]\label{def:plan}
We say that two interactions $\alpha$ and $\beta$, $\alpha \neq \beta$, \emph{conflicts} if $\p{\alpha}\cap\p{\beta}\neq\emptyset$, and we write $\alpha\#\beta$.
A plan $\pi$ is a partial function $\pi:\gamma \to \mathbb{R}_{\ge 0}$ defining relative times for 
executing a subset of non conflicting interactions, i.e.:
\begin{displaymath}
  \alpha\neq\alpha',\pi(\alpha)\neq\perp,\pi(\alpha')\neq\perp\implies \neg(\alpha\#\alpha').
\end{displaymath}
We also denote by $\confl(\pi)$ the set of interactions conflicting with the plan $\pi$, i.e. $\confl(\pi) = \{ \alpha \ | \ \exists \beta \# \alpha \ . \ \pi (\beta) \neq \bot \}$, and $\p{\pi}$ the set of components involved in interactions planned by $\pi$, i.e. $\p{\pi} = \{ B_i \ | \ \exists \alpha \ . \ \pi(\alpha) \neq \bot \wedge B_i \in \p{\alpha} \}$.
\end{definition}
%We write $(\alpha,\delta)\in\pi$ the planning of interaction $\alpha$ in $\delta$ units of 
We denote by $\min\pi$ the closest relative execution time of interactions in the plan $\pi$, i.e. $\min\pi = \textnormal{ min } \{ \pi(\alpha) \ | \ \alpha \in \gamma \wedge \pi(\alpha) \neq \bot \} \cup \{ +\infty \}$.
Notice that since $\pi$ stores relative times, whenever time progresses by $\delta$ the value $\pi(\alpha)$ assigned by $\pi$ to an interaction $\alpha$ should be decreased by $\delta$, until it reaches $0$ which means that $\alpha$ have to execute.
We write $\pi-\delta$ describing the progress of time 
over the plan, that is, $(\pi-\delta)(\alpha) = \pi(\alpha) - \delta$ for interactions $\alpha$ such that $\pi(\alpha) \neq \bot$.
We also write
$\pi-\alpha$ to denote the removal of interaction $\alpha$ from the plan $\pi$, i.e. $(\pi-\alpha)(\beta) = \pi(\beta)$ for any interaction $\beta \neq \alpha$, $(\pi-\alpha)(\alpha) = \bot$.
Similarly, $\pi \cup \{ \alpha \mapsto \delta \}$ assigns relative time $\delta$ to $\alpha$, $\alpha \notin conf(\pi)$, into existing plan $\pi$, i.e. $(\pi \cup \{ \alpha \mapsto \delta \})(\beta) = \delta$ for $\beta = \alpha$, $(\pi \cup \{ \alpha \mapsto \delta \})(\beta) = \pi(\alpha)$ otherwise.
Finally, the plan $\pi$ such that $\pi(\alpha) = \bot$ for all interactions $\alpha \in \gamma$ is denoted by $\emptyset$.

We define below the semantics for planning each interaction $\alpha\in\gamma$ with $\deltama$-horizon.

\begin{definition}[Weak Planning Semantics]\label{def:plan}
Given a set of components\\ $\{B_1,\cdots,B_n\}$ and an interaction set $\gamma$,
we define the weak planning semantics of the composite component $S = (\Loc,\loc_0,\gamma,T_{\gamma},\X,\inv)$,
as the labeled transition system $S_p=(\Q_{\pi},
\gamma\cup\realpos\cup\{\plana\},\tranbp{}{3})$ where:
\begin{itemize}
  \item $\Q_{\pi}=\Loc\times\mathcal{V}(\X)\times\Pi$, where $\Loc$ is the set of global location,
    $\mathcal{V}(\X)$ is the set of global clocks valuations, and $\Pi$ is the set of plans.
  \item $\plana$ defines the action of planning interactions
  \item $\tranbp{}{3}$ is the set of labeled transitions defined by the rules:
  \begin{itemize}
    \item Plan:
    \begin{align*}
      &\delta\le\deltama,\alpha\in\gamma,\p{\alpha}\cap\p{\pi}=\emptyset\quad\enabledfrom(\alpha)\\
     \cline{1-2}
     &\hspace{1.5cm}(\loc,\val,\pi)\tranbp{\plan{\alpha,\delta}}{4}(\loc,\val,\pi\cup\{\alpha\mapsto\delta\}).\\
    \end{align*}
    \vspace*{-10mm}

    \item Exec:
     \begin{align*}
       &\hspace{12mm}\pi(\alpha)=0\\
            \cline{1-2}
          &(\loc,\val,\pi)\tranbp{\alpha}{3}(\loc',\val',\pi-\alpha)\}\\
        \end{align*}
  \item Time Progress: $\delta\in\realpos$
      \begin{align*}
        &\delta\le \text{min } \pi
        \wedge\tpc{i}(\val_i+\delta)_{i\in\{1,\cdots,n\}}\\
        \cline{1-2}
        &\hspace{1cm}(\loc,\val,\pi)\tranbp{\delta}{3}(\loc,\val+\delta,\pi-\delta)\\
          \end{align*}
  \end{itemize}
  \end{itemize}
\end{definition}
\begin{example}
  \label{exp:dl}
  Let us consider the following execution sequence for the example of Figure~\ref{fig:run} under the weak planning semantics rules and 
  for a value $\deltamax=5$ for all interactions except $\alpha_5$ and $\alpha_6$ that will be assigned a $\deltamax=3$:
      \begin{displaymath}
      %  \scriptsize{
        \small{
        \begin{split}
          &((\loc_0^1,\loc_0^2,\loc_0^3,\loc_0^4),(0,0,0),\emptyset)\tranbp{\plan{\alpha_1,5}}{6}((\loc_0^1,\loc_0^2,\loc_0^3,\loc_0^4),(0,0,0),\{\alpha_1\mapsto5\})\tranbp{5}{3}\\&
          ((\loc_0^1,\loc_0^2,\loc_0^3,\loc_0^4),(5,5,5),\{\alpha_1\mapsto0\})\tranbp{\alpha_1}{3}((\loc_1^1,\loc_1^2,\loc_0^3,\loc_0^4),(5,5,5),\emptyset)\tranbp{\plan{\alpha_3,2}}{6}
          \\&((\loc_1^1,\loc_1^2,\loc_0^3,\loc_0^4),(5,5,5),\{\alpha_3\mapsto2\})\tranbp{2}{3}((\loc_1^1,\loc_1^2,\loc_0^3,\loc_0^4),(7,7,7),\{\alpha_3\mapsto0\})\tranbp{\alpha_3}{3}
          \\&((\loc_0^1,\loc_2^2,\loc_0^3,\loc_0^4),(0,7,0),\emptyset)\tranbp{\plan{\alpha_5,2}}{6}((\loc_0^1,\loc_2^2,\loc_0^3,\loc_0^4),(0,7,0),\{\alpha_5\mapsto2\})\tranbp{2}{3}
          \\&((\loc_0^1,\loc_2^2,\loc_0^3,\loc_0^4),(2,9,2),\{\alpha_5\mapsto0\})\tranbp{\alpha_5}{3}((\loc_0^1,\loc_3^2,\loc_0^3,\loc_1^4),(2,9,2),\emptyset)\tranbp{\plan{\alpha_2,3}}{6}\\
            &((\loc_0^1,\loc_3^2,\loc_0^3,\loc_1^4),(2,9,2),\{\alpha_2\mapsto3\})\tranbp{3}{3}((\loc_0^1,\loc_3^2,\loc_0^3,\loc_1^4),(5,12,5),\{\alpha_2\mapsto0\})\tranbp{\alpha_2}{3}\\
            &((\loc_1^1,\loc_3^2,\loc_1^3,\loc_1^4),(5,12,5),\emptyset)\tranbp{\plan{\alpha_4,0}}{6}((\loc_1^1,\loc_3^2,\loc_1^3,\loc_1^4),(5,12,5)\{\alpha_4\mapsto0\})\tranbp{\alpha_4}{3}\\
            &((\loc_0^1,\loc_3^2,\loc_2^3,\loc_1^4),(5,0,0),\emptyset)\tranbp{\plan{\alpha_7,4}}{6}((\loc_0^1,\loc_3^2,\loc_2^3,\loc_1^4),(5,0,0),\{\alpha_7\mapsto4\})\tranbp{3}{3}\\
            &((\loc_0^1,\loc_3^2,\loc_2^3,\loc_1^4),(8,3,3),\{\alpha_7\mapsto1\})
        \end{split}
      }
      \end{displaymath}
This execution sequence represents a path that alternates plan actions, time steps and execution of some interactions. We can see that for interaction
$\alpha_7$ which is planned 4 units of time ahead, the system cannot reach the state from which it can be executed since there is a time progress expiration
in component $T_2$ after 3 time units from planning this interaction. This means that local planning of interactions doesn't always allow the progress of time and may
thus, introduce deadlocks even if the system under the global semantics rules is deadlock-free. 
      
\end{example}

\subsection{Relation between Global and Weak Planning Semantics}
We use weak simulation to compare the model under
the global semantics rules and the one under the weak planning semantics rules
by considering \plana-transitions unobservable.
As explained in Example~\ref{exp:dl}, the weak planning semantics does not preserve the deadlock property of our system.
Nevertheless, the following proves weak simulation relations between the two semantics.

\begin{theorem}\label{thm:pienf}
  For all the reachable states $(\loc,\val,\pi)$ of the weak planning semantics, and $\forall\alpha\in\pi$, the
  predicate $\enabledfrombis{\pi(\alpha)}(\alpha)$ is $\true$.
\end{theorem}

Let $S_g=(\Q_g,\gamma\cup\realpos,\transit{}_{\gamma})$ (resp. $S_{p}=(\Q_{p},\gamma\cup\realpos\cup\{\plana\},\tranbp{}{3})$)
the labeled transition system characterizing the global (resp. weak planning) semantics. 
\begin{proposition}\mbox{}\\
  \vspace{-6mm}
  \begin{description}[labelwidth=1.5cm]
    \item[\namedlabel{itm:1}{Relation 1}]$\forall\delta\in\realpos.(\loc,\val,\pi)\tranbp{\delta}{3}(\loc',\val',\pi')
      \Rightarrow (\loc,\val)\transit{\delta}_{\gamma}(\loc',\val')$
    \item[\namedlabel{itm:2}{Relation 2}]$\forall\alpha\in\gamma.(\loc,\val,\pi)\tranbp{\alpha}{3}(\loc',\val',\pi')
      \Rightarrow (\loc,\val)\transit{\alpha}_{\gamma}(\loc',\val')$
  \end{description}
\end{proposition}
It is straightforward that~\ref{itm:1} is a consequence of the definition of time progress in the weak planning semantics. For Relation~\ref{itm:2},  
using Definition~\ref{def:plan}, we can deduce that:\\
\begin{displaymath}
(\loc,\val,\pi)\tranbp{\alpha}{3}(\loc',\val',\pi')\Rightarrow \pi(\alpha)=0,
\end{displaymath}
By Theorem~\ref{thm:pienf}, this implies that $\enabledfrombis{0}(\alpha)$ is $\true$ at state $(\loc,\val,\pi)$, meaning that $\enabled(\alpha)$ is
also $\true$, which allows to infer~\ref{itm:2}.  

\begin{corollary}\label{cr:reach}
  If a state $(\loc,\val,\pi)\in\reach(S_p)$, then $(\loc,\val)\in\reach(S_g)$.
\end{corollary}

\begin{definition}[Weak Simulation]
  A weak simulation over $A=(\Q_A,\sum\cup\{\beta\},\to_A)$ and $B=(\Q_B,\sum\cup\{\beta\},
  \to_B)$ is a relation $R\subseteq \Q_A\times \Q_B$ such that we have: 
  $\forall(q,r)\in R, a\in \sum .q\transit{a}_A q' \implies\exists r':(q',r')\in R\wedge r\transit{
  \beta^*a\beta^*}_B r' \text{ and } \forall(q,r)\in R: q\transit{\beta}_Aq'\implies\exists r':
  (q',r')\in R\wedge r\transit{\beta^*}r'$.
  B simulates A, denoted by $A\simu{R}B$, means that B can do everything A does.
\end{definition}
The definition of weak simulation is based on the unobservability of $\beta-$transitions. In our case, $\beta-$transitions corresponds to $\plana-$transitions.
\begin{corollary}\label{cr:sim}
  $S_p\simu{R_1} S_g$ with $R_1=\{(\q,\pi);\q)\in\Q_p\times\Q_g\}$.
\end{corollary}

Corollary~\ref{cr:sim} corresponds to a notion of correctness of the weak planning semantics: any execution in weak planning semantics corresponds to an execution in the global state semantics.

\begin{theorem}
  \label{thm:sim1}
  $S_g\simu{R_2} S_p$ with $R_2=\{(\q;(\q,\pi))\in\Q_g\times\Q_p| \pi=\emptyset\}$. 
\end{theorem}

Theorem~\ref{thm:sim1} states that the weak planning semantics preserves all execution sequences of the global state semantics.
They are obtained using immediate planning, i.e. plans $\pi$ such that $\pi(\alpha) = 0$ or $\pi(\alpha) = \bot$.
The weak planning semantics aims to reduces the impact of communication delays in the system through planning interactions execution ahead, and by
considering only the state of components involved in the planned interaction, which is more suitable for 
distributed real-time systems than the global state semantics.
It does not restrict the behavior of the global state semantics (see Theorem~\ref{thm:sim1}), and it executes ony sequences allowed by the global state semantics (see Corollary~\ref{cr:sim}).
However, it may introduce deadlocks as shown by the scenario presented in Example~\ref{exp:dl}.
In the following, we present sufficient conditions for deadlock-free 
planning of interactions.

