\section{Deadlock-free Planning}\label{sec:3}
\vspace*{-1mm} As explained in Example~\ref{exp:dl}, local planning of interactions can 
  introduce deadlocks in the system since it does not consider time progress
  conditions of components not participating in the planned interactions. Effectively,
  the weak planning semantics ensures that time can progress until the chosen execution date only w.r.t
  timing constraints of participating components, but such progress may be disallowed by the rest of the system leading to deadlock states.
  In this section, we provide sufficient conditions for having deadlock-free planning.

  Planning an interaction $\alpha$ implies not only blocking components 
  participating in $\alpha$ until $\alpha$ executes, but also preventing the system from planning interactions
  involving these components, that is, interactions of $\confl(\alpha)$.
  Consequently, the subset of interactions $\gamma' \subseteq \gamma$ that can be planned at a given state $(\loc,\val,\pi)$ depends on the content of the plan $\pi$.
  It satisfies $\gamma'=\{\gamma\setminus\pi\cup\confl(\pi)\}$.

  By Corollary~\ref{cr:reach}, a (reachable) deadlock state $(\loc,\val,\pi)$ of the weak planning semantics $S_p$ is such that $(\loc,\val)$ is a reachable state of the global state semantics $S_g$.
  Since we assume that $S_g$ is deadlock-free, $(\loc,\val)$ is not a deadlock in $S_g$.
  A deadlock state $(\loc,\val,\pi)$ of $S_p$ is caused by the plan $\pi$ which
  is restricting the execution in $S_p$ w.r.t. $S_g$: interactions $\alpha$ of $\pi$ cannot execute before $\pi(\alpha)$ time units, and interactions $\alpha \in \confl(\pi)$ are blocked for (at least)
  $\textnormal{max } \{ \pi(\beta) \ | \ \beta\#\alpha \}$.
  Notice that due to well-formed guards, in a deadlock state $(\loc,\val,\pi)$ we have necessarily $\at{\loc_i} \wedge \urg(\tpc{\loc_i})$ for a location $\loc_i$ of a component $B_i \notin \p{\pi}.$ 

  \begin{theorem}\label{thm:dlp}
    If a state $(\loc,\val,\pi)\in\reach(S_p)$ deadlocks, the following equation is satisfied:
  \begin{equation}\label{eq:dlp}
  \begin{split}
    &\underbrace{\bigwedge_{\alpha\in\pi}\enabledfrombis{\pi(\alpha)}(\alpha)}_\text{\mytag{A}{A}}\quad\wedge
    \underbrace{\bigvee_{B_i\in S\setminus\p{\pi}}\bigvee_{\loc_i\in\Loc_i} \at{\loc_i} \wedge \urg(\tpc{\loc_i})}_\text{\mytag{B}{B}}\\
    &\wedge\underbrace{\bigwedge_{\alpha\in\pi}\pi(\alpha)\neq 0\wedge\big(\bigvee_{\alpha\in\pi}(\enabled(\alpha)\vee
    \bigvee_{\alpha\in\confl(\pi)}\enabled(\alpha)\big)}_\text{\mytag{C}{C}}
  \end{split}
  \end{equation}
  \end{theorem}
  From Theorem~\ref{thm:pienf}, Term~\ref{A} of Equation~\ref{eq:dlp} represents an invariant of the system. On the other hand, terms~\ref{B} and
  ~\ref{C} characterize the deadlock: Term~\ref{B} expresses the urgency of time progress condition in components not involved in the planned interactions, 
  whereas, term~\ref{C} specifies the origin of the deadlock: it characterizes states $(\loc,\val,\pi)$ of $S_p$ for which $\pi$ restricts the execution of an 
  interaction $\alpha$ whereas it can be executed at $(\loc,\val)$ in $S_g$.
  As explained above, such an interaction satisfies $\pi(\alpha) > 0$ or $\alpha \in \confl(\pi)$.
  
  It is clear that Equation~\ref{eq:dlp} depends on the reachable states of the planning semantics since it explicitly depends on plans $\pi$.
  The following gives weaker conditions for deadlocks which are independent of the plan. 
  \begin{theorem}\label{thm:phi}
    Let $\Phi(\alpha)$ be the following predicate:
    \begin{equation}\label{eq:phi}
    %\begin{split}
      \bigfrown{\enabledfromd(\alpha)}\wedge
      \bigvee_{B_i\in S\setminus\p{\alpha}}\bigvee_{\loc_i\in\Loc_i}\at{\loc_i}\wedge\urg(\tpc{\loc_i})
    \wedge\bigvee_{\beta\in\alpha\cup\confl(\alpha)}\enabled(\beta)
    %\end{split}
  \end{equation}
  where $\bigfrown{\enabledfromd(\alpha)}$ is the result of transforming all the timing constraints of the form $x\le ct$ by $x<ct$
  in $\backwardd(\bigwedge_{a_i\in\alpha}\guard{a_i}{\loc_i})$ of $\enabledfromd(\alpha)$.
  
  If a reachable state of the system $(\loc,\val,\pi)$ deadlocks then the following is satisfied:
  \begin{equation}\label{eq:dlpf}
      \exists\alpha\in\gamma, \Phi(\alpha)\wedge\deltama\neq0
  \end{equation}
  \end{theorem}
 
  Let $\sched(\alpha,\deltama)$ be the following predicate:
  \begin{displaymath}
    \sched(\alpha,\deltama) = \neg\Phi(\alpha)\vee(\deltama=0)
  \end{displaymath}

  Using Theorem~\ref{thm:phi} and Corollary~\ref{cr:reach}, we can conclude that for all interactions $\alpha\in\gamma$ and for all reachable states of the global state semantics $S_g$,
  if the predicate $\sched(\alpha,\deltama)$ is satisfied, then the weak planning semantics is deadlock-free.
  Notice that given an interaction $\alpha \in \gamma$ the satisfaction of $\sched(\alpha,\deltama)$ on $Reach(S_g)$ depends only on $\deltama$.
  Moreover, it is monotonic, that is, if it holds for $\deltama$ then it holds for any $\deltama' < \deltama$.
  This provides means for building implementations that plan interactions as soon as possible by taking for $\deltama$ the maximal value of $\delta$ such that $\sched(\alpha,\delta)$ holds on $Reach(S_g)$.
  
