\section{Implementation and Experiments}\label{sec:5}

The presented method has been implemented as a middleend filter of the BIP compiler. BIP~\cite{Basu:BIP} is a highly expressive, component-based
framework with rigorous semantics that allows the construction of complex, hierarchically structured models from single components characterized by their behavior.
The method input consists of real-time BIP model
and a file containing an approximation of the reachable states of components combined with history clock inequalities as explained in 
Section~\ref{sec:4}. The latter is generated using the RTD-Finder tool, a verification tool for real-time component based
systems modeled in the RT-BIP language. Our filter generates for each interaction of the input model a Yices~\cite{yices} file containing system invariants together with the condition for
planning the considered interaction, that is, $\neg\sched(\alpha,\deltama)$. Thereafter, Yices checks the satisfiability of
$\bigwedge_{i=1}^n \reach(B_i)\wedge\quad\historyineq(S_g)\quad\wedge\quad\neg\sched(\alpha,\deltama)$. We also define $\deltama$ as free variable.
If this condition is unsatisfiable, then planning interactions
$\alpha$ is safe and unbounded that is, $\deltamax=+\infty$. Otherwise, Yices generates a counter-example. Due to the monotony of the condition, this counter-example can be
used to find the maximal value of $\deltama$  satisfying the above condition using a binary search algorithm.
Together, the determined values of the bounds $\deltamax$ for each interaction will affect the dynamic of the hole system: for an interaction 
$\alpha$ the greater $\deltama$ is, the more flexible the scheduling of $\alpha$ will be. 

We ran our experiments on three other models besides of the model presented in Figure~\ref{fig:run}: Pacemaker~\cite{pace}, Fischer~\cite{fisch}, and
Gear controller~\cite{gear}. We developed an implementation of these models in RT-BIP.
The following tables show the result of our experiments. Table~\ref{t:1} gives a detailed result of the experiments ran on 
the Task Manager model~\ref{fig:run}. It summarizes, for each interaction, its \emph{conflicting interactions} and the potential
locations for which a time progress condition may expire while planning it (column $\inv$). The last column, $\deltama$, details the maximum horizon for 
planning interaction $\alpha$. Notice that the symmetry of the model allows to perform the verification on interactions $\alpha_1,\alpha_3,\alpha_5\text{, and }
\alpha_7$ and deduce the results for the other interactions.
Table~\ref{t:2} depicts the results of our experiments on different models. For each model, it summarizes the number of interactions
that can be safely planned with an unbounded horizon ($\deltamax=\infty$). It also gives the number of interactions that cannot be planned in advance, and thus,
need to be executed immediately after being planned $(\deltamax=0)$.
   \vspace{-.8cm}
   \begin{table*}
     \caption{Detailed Results of the Task Manager Experiments}\label{t:1}
    \centering
    \vspace*{3mm}
    \begin{tabular}{| c || c | c | c | c | c |}
      \hline
      Interaction      &  Conflicting Interactions     & $\inv$       & $\deltama$  \\ \hline
      $\alpha_1$       & $\alpha_2,\alpha_4,\alpha_8$  & $\loc^3_2$   & $\infty$ \\\hline
      $\alpha_3$       & $\alpha_2,\alpha_4,\alpha_8$  & $\loc^3_2$   & $\infty$ \\\hline
      $\alpha_5$       & $\alpha_6,\alpha_8$           & $\loc^3_2$   & $\infty$\\\hline
      $\alpha_7$       & $\alpha_2,\alpha_4$           & $\loc^3_2$   & 0 \\\hline
    \end{tabular}
   \end{table*}
   \vspace{-1.5cm}
   \begin{table*}
     \caption{Results of Experiments}\label{t:2}
    \vspace*{3mm}
    \centering
    \begin{tabular}{| c || c | c  | c |}
      \hline
      \multirow{2}{*}{Model}             & \multicolumn{3}{c|}{Number of Interactions}\\\cline{2-4}
                                         & $\deltamax=0$ &$\deltamax=\infty$ & total\\\hline
      
      Task Manager                       &  2           & 6                & 8 \\\hline 
      Pacemaker                          &  0           & 6                & 6  \\ \hline
      Gear                               &  0           & 17               & 17 \\ \hline
      Fischer                            & 0            & 10               & 10 \\ \hline
    \end{tabular}
   \end{table*}
   \vspace{-1cm}
\section{Conclusion and Future Work}

We presented a method for scheduling real-time systems in a distributed context considering models including multiparty interactions.
The proposed approach defines sufficient conditions ensuring a deadlock-free local planning of interactions with certain horizons.
Moreover, it is proved that those conditions are interaction dependent, in other terms, this means that changing the planning horizon of an interaction
does not affect the planning of other interactions. A key innovative idea is the use of global knowledge in addition to local components informations to enhance
the local scheduling of interactions. The computed knowledge captures not only the way components synchronize through interactions, but it also 
consider the history clock inequalities between those interactions and express explicitly the synchrony of time progress. 

There are many open problems to be investigated such as: \emph{(i)} when planning an interaction, identifying conditions based on
the state of components involved in this interaction, and \emph{(ii)} defining a lower bound for planning interaction. The latter represents an important point
meaning that, if planning interactions can be ensured for a lower bound, that effectively represents the communication delays of the target platform, then all the problems
induced by those delays, such as global consistency and performance dropping will be solved. 
