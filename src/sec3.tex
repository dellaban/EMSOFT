\section{Local Planning of Interactions}
\label{sec3}
To the best of our knowledge, distributed platforms rarely offer built-in primitives
for the high level coordination, required by models where different components
need to synchronize together. Thus, the implementation of such models requires
an execution engine (for centralized execution), or more (for a more distributed execution),
responsible of coordinating components synchronizations using simpler primitives,
such as point-to-point messages passing (cite bip parrallel real time) and maybe a figure.

However, the semantics presented in previous section does not distinguish between 
an engine decision time and the actual execution time of components.
Particularly, communication delays induced by distributed platforms may introduce 
deadline misses during this process.
Additionally, Section~\ref{sec2} semantics is based on the global state of the system,
that is, the operational semantics rules is achieved through global states, which
break the principle of distribution where interaction execution should require only 
knowing the state of its participating components. 
We introduced in (cite FM) the \emph{weak planning semantics}, a semantics for local
planning of interaction within a defined upper bounded horizon. This semantics plans interaction based only on
the state of its involved components. This approach aim to differentiate the scheduling decision 
time of an interaction and its actual execution. In this paper we extend this work to lower bounded
horizons, which effectively, represent the estimation of the target platform communication
delays. In what follows, we first give some formal definitions, then we present our extension of
the weak planning semantics and discuss its properties.

\subsubsection*{Preliminaries}\label{subsec:wp}\mbox{}\\ 
We define the predicate $\plnIntxt{\alpha}{\delta}$ characterizing all states
  from which $\alpha$ can be planned and execute after $\delta$ units of time, that is, 
  $\alpha$ will be \emph{enabled} after a time progresses of $\delta$ units of time:
  \begin{equation}\label{eq:enf}
    \plnIn{\alpha}{\delta}
\end{equation}
%Notice that for $(\loc,\val),(\loc',\val')\in\reach(S)$, and $a_i\in\alpha$ we have:
%$$\enabledfrom(\alpha) \text{ at }(\loc,\val)\Rightarrow \enabled(\alpha) \text{ at } (\loc',\val'),\text{ such that }
%\loc_i=\loc'_i \text{ and } \val_i=\val'_i.$$

\begin{property}\label{pt:plnIn1}
  Let $(\loc,\val)$ be a state of the composition $S$. For any interaction $\beta\in\gamma$ such that, $\p{\alpha}\cap\p{\beta}=\emptyset$
  and $(\loc,\val)\transit{\beta}_{\gamma}(\loc',\val')$, where $\p{\alpha}$ (resp. $\p{\beta}$) represents components participating in 
  interaction $\alpha$ (resp. $\beta$), if $\plnIntxt{\alpha}{\delta}$ holds at state $(\loc,\val)$ then it still holds at state $(\loc',\val')$.
\end{property}
This property derives from the fact that executing interactions with disjoint set of components than $\alpha$ does not change the states
of components participating in $\alpha$, that is, for $a_i\in\alpha$ we have $\loc_i=\loc'_i$ and $\val_i=\val'_i$.


\begin{property}\label{pt:plnIn2}
  Let $(\loc,\val)$ and $(\loc,\val+\delta')$, with $\delta'\in\realpos$ be two states of the composition $S$. 
  If $\plnIntxt{\alpha}{\delta}$ is $\true$ at state $(\loc,\val)$ then $\plnIntxt{\alpha}{\delta-\delta'}$ is true at state
  $(\loc,\val+\delta')$ for $\delta'\le\delta$.
\end{property}
This property can be found directly by writing Equation~\ref{eq:enf} on state $(\loc,\val+\delta')$.

Let $\tcal{H}$ be a partial function $\tcal{H}: \gamma \to\realpos\times\realpos$ that defines for each
interaction $\alpha\in\gamma$, its respective planning horizons as an interval $[\hmn,\hmx]$. 
We define the predicate $\plntxt{\alpha}$ characterizing all states from which 
$\alpha$ can be planned w.r.t its planning horizons as follows:  
\begin{displaymath}
  \pln{\alpha}
\end{displaymath}
  with $\backhtxt$ represents an adaptation of the backward operators~\cite{tripakis98:thesis} that satisfies:
\begin{displaymath}
\backh
\end{displaymath}
\begin{property}\label{pt:pln}
  If the predicate $\plnIntxt{\alpha}{\delta}$ is $\true$ at a state $(\loc,\val)$, then the 
  predicate $\plntxt{\alpha}$ is also $\true$ for $\delta\in[\hmn,\hmx]$.
\end{property}

\begin{definition}[Plan]\label{def:plan}
We say that two interactions $\alpha$ and $\beta$, $\alpha \neq \beta$, \emph{conflicts} if $\p{\alpha}\cap\p{\beta}\neq\emptyset$, and we write $\alpha\#\beta$.
A plan $\pi$ is a partial function $\pi:\gamma \to\realpos$ defining relative times for 
executing a subset of non conflicting interactions, i.e.:
\begin{displaymath}
  \alpha\neq\alpha',\pi(\alpha)\neq\perp,\pi(\alpha')\neq\perp\implies \neg(\alpha\#\alpha').
\end{displaymath}
We also denote by $\confl(\pi)$ the set of interactions conflicting with the plan $\pi$, i.e. $\confl(\pi) = \{ \alpha \ | \ \exists \beta \# \alpha \ . \ \pi (\beta) \neq \bot \}$, and $\p{\pi}$ the set of components involved in interactions planned by $\pi$, i.e. $\p{\pi} = \{ B_i \ | \ \exists \alpha \ . \ \pi(\alpha) \neq \bot \wedge B_i \in \p{\alpha} \}$.
\end{definition}
%We write $(\alpha,\delta)\in\pi$ the planning of interaction $\alpha$ in $\delta$ units of 
We denote by $next \ \pi$ the closest relative execution time of interactions in the plan $\pi$, i.e. $next \ \pi = \textnormal{ min } \{ \pi(\alpha) \ | \ \alpha \in \gamma \wedge \pi(\alpha) \neq \bot \} \cup \{ +\infty \}$.
Notice that since $\pi$ stores relative times, whenever time progresses by $\delta$ the value $\pi(\alpha)$ assigned by $\pi$ to an interaction $\alpha$ should be decreased by $\delta$, until it reaches $0$ which means that $\alpha$ have to execute.
We write $\pi-\delta$ describing the progress of time 
over the plan, that is, $(\pi-\delta)(\alpha) = \pi(\alpha) - \delta$ for interactions $\alpha$ such that $\pi(\alpha) \neq \bot$.
We also write
$\pi-\alpha$ to denote the removal of interaction $\alpha$ from the plan $\pi$, i.e. $(\pi-\alpha)(\beta) = \pi(\beta)$ for any interaction $\beta \neq \alpha$, $(\pi-\alpha)(\alpha) = \bot$.
Similarly, $\pi \cup \{ \alpha \mapsto \delta \}$ assigns relative time $\delta$ to $\alpha$, $\alpha \notin conf(\pi)$, into existing plan $\pi$, i.e. $(\pi \cup \{ \alpha \mapsto \delta \})(\beta) = \delta$ for $\beta = \alpha$, $(\pi \cup \{ \alpha \mapsto \delta \})(\beta) = \pi(\beta)$ otherwise.
Finally, the plan $\pi$ such that $\pi(\alpha) = \bot$ for all interactions $\alpha \in \gamma$ is denoted by $\emptyset$.

We define below the semantics for planning each interaction $\alpha\in\gamma$ with $\delta$-horizon within $[\hmn,\hmx]$.

\begin{definition}[Planning Semantics]\label{def:pln_sem}
Given a set of components $\{B_1,\cdots,B_n\}$ and an interaction set $\gamma$,
we define the planning semantics of the composite component $S = (\Loc,\loc_0,\gamma,T_{\gamma},\X,\inv)$,
as the labeled transition system $S_p=(\Q_p,
\gamma\cup\realpos\cup\{\plana\},\tranbp{}{3})$ where:
\begin{itemize}
  \item $\Q_p=\Loc\times\mathcal{V}(\X)\times\Pi$, where $\Loc$ is the set of global location,
    $\mathcal{V}(\X)$ is the set of global clocks valuations, and $\Pi$ is the set of plans.
  \item $\plana$ defines the action of planning interactions
  \item $\tranbp{}{3}$ is the set of labeled transitions defined by the rules:
  \begin{itemize}
    \item Plan: $\alpha\in\gamma,\delta\in[\hmn,\hmx]$
    \begin{align*}
      &\hspace{4mm}\alpha\notin\confl(\pi)\wedge\plnIntxt{\alpha}{\delta}\\
     \cline{1-2}
     &(\loc,\val,\pi)\tranbp{\plan{\alpha,\delta}}{4}(\loc,\val,\pi\cup\{\alpha\mapsto\delta\}).\\
    \end{align*}
    \vspace*{-10mm}

    \item Exec: $\alpha\in\gamma$
     \begin{align*}
       &\hspace{14mm}\pi(\alpha)=0\\
            \cline{1-2}
          &(\loc,\val,\pi)\tranbp{\alpha}{3}(\loc',\val',\pi-\alpha)\}\\
        \end{align*}
  \item Time Progress: $\delta\in\realpos$
      \begin{align*}
        &\delta\le \text{next } \pi
        \wedge\tpc{i}(\val_i+\delta)_{i\in\{1,\cdots,n\}}\\
        \cline{1-2}
        &\hspace{4mm}(\loc,\val,\pi)\tranbp{\delta}{3}(\loc,\val+\delta,\pi-\delta)\\
          \end{align*}
  \end{itemize}
  \end{itemize}
\end{definition}
\begin{example}
  \label{exp:dl}
  Let us consider the following execution sequence for the example of Figure~\ref{fig:run} under the weak planning semantics rules and 
  for a value $\deltamax=5$ for all interactions except $\alpha_5$ and $\alpha_6$ that will be assigned a $\deltamax=3$:
      \begin{displaymath}
      %  \scriptsize{
        \small{
        \begin{split}
          &((\loc_0^1,\loc_0^2,\loc_0^3,\loc_0^4),(0,0,0),\emptyset)\tranbp{\plan{\alpha_1,5}}{6}((\loc_0^1,\loc_0^2,\loc_0^3,\loc_0^4),(0,0,0),\{\alpha_1\mapsto5\})\tranbp{5}{3}\\&
          ((\loc_0^1,\loc_0^2,\loc_0^3,\loc_0^4),(5,5,5),\{\alpha_1\mapsto0\})\tranbp{\alpha_1}{3}((\loc_1^1,\loc_1^2,\loc_0^3,\loc_0^4),(5,5,5),\emptyset)\tranbp{\plan{\alpha_3,2}}{6}
          \\&((\loc_1^1,\loc_1^2,\loc_0^3,\loc_0^4),(5,5,5),\{\alpha_3\mapsto2\})\tranbp{2}{3}((\loc_1^1,\loc_1^2,\loc_0^3,\loc_0^4),(7,7,7),\{\alpha_3\mapsto0\})\tranbp{\alpha_3}{3}
          \\&((\loc_0^1,\loc_2^2,\loc_0^3,\loc_0^4),(0,7,0),\emptyset)\tranbp{\plan{\alpha_5,2}}{6}((\loc_0^1,\loc_2^2,\loc_0^3,\loc_0^4),(0,7,0),\{\alpha_5\mapsto2\})\tranbp{2}{3}
          \\&((\loc_0^1,\loc_2^2,\loc_0^3,\loc_0^4),(2,9,2),\{\alpha_5\mapsto0\})\tranbp{\alpha_5}{3}((\loc_0^1,\loc_3^2,\loc_0^3,\loc_1^4),(2,9,2),\emptyset)\tranbp{\plan{\alpha_2,3}}{6}\\
            &((\loc_0^1,\loc_3^2,\loc_0^3,\loc_1^4),(2,9,2),\{\alpha_2\mapsto3\})\tranbp{3}{3}((\loc_0^1,\loc_3^2,\loc_0^3,\loc_1^4),(5,12,5),\{\alpha_2\mapsto0\})\tranbp{\alpha_2}{3}\\
            &((\loc_1^1,\loc_3^2,\loc_1^3,\loc_1^4),(5,12,5),\emptyset)\tranbp{\plan{\alpha_4,0}}{6}((\loc_1^1,\loc_3^2,\loc_1^3,\loc_1^4),(5,12,5)\{\alpha_4\mapsto0\})\tranbp{\alpha_4}{3}\\
            &((\loc_0^1,\loc_3^2,\loc_2^3,\loc_1^4),(5,0,0),\emptyset)\tranbp{\plan{\alpha_7,4}}{6}((\loc_0^1,\loc_3^2,\loc_2^3,\loc_1^4),(5,0,0),\{\alpha_7\mapsto4\})\tranbp{3}{3}\\
            &((\loc_0^1,\loc_3^2,\loc_2^3,\loc_1^4),(8,3,3),\{\alpha_7\mapsto1\})
        \end{split}
      }
      \end{displaymath}
This execution sequence represents a path that alternates plan actions, time steps and execution of some interactions. We can see that for interaction
$\alpha_7$ which is planned 4 units of time ahead, the system cannot reach the state from which it can be executed since there is a time progress expiration
in component $T_2$ after 3 time units from planning this interaction. This means that local planning of interactions doesn't always allow the progress of time and may
thus, introduce deadlocks even if the system under the global semantics rules is deadlock-free. 
      
\end{example}

\subsection{Relation between Global and Weak Planning Semantics}
We use weak simulation to compare the model under
the global state semantics rules and the one under the planning semantics rules
by considering \plana-transitions unobservable.
As explained in Example~\ref{exp:dl}, the planning semantics does not preserve the deadlock freedom property of our system.
Nevertheless, the following proves weak simulation relations between the two semantics.

\begin{theorem}\label{thm:pi_pln}
  For all the reachable states $(\loc,\val,\pi)$ of the planning semantics, and $\forall\alpha\in\pi$, the
  predicate $\plnIntxt{\alpha}{\pi(\alpha)}$ is $\true$.
\end{theorem}

Let $S_g=(\Q_g,\gamma\cup\realpos,\transit{}_{\gamma})$ (resp. $S_{p}=(\Q_{p},\gamma\cup\realpos\cup\{\plana\},\tranbp{}{3})$)
the labeled transition system characterizing the global (resp. planning) semantics. 
\begin{proposition}\mbox{}\\
  \label{prop:relation}
  \vspace{-6mm}
  \begin{description}[labelwidth=1.5cm]
    \item[\namedlabel{itm:1}{Relation 1}]$\forall\delta\in\realpos.(\loc,\val,\pi)\tranbp{\delta}{3}(\loc',\val',\pi')
      \Rightarrow (\loc,\val)\transit{\delta}_{\gamma}(\loc',\val')$
    \item[\namedlabel{itm:2}{Relation 2}]$\forall\alpha\in\gamma.(\loc,\val,\pi)\tranbp{\alpha}{3}(\loc',\val',\pi')
      \Rightarrow (\loc,\val)\transit{\alpha}_{\gamma}(\loc',\val')$
  \end{description}
\end{proposition}
It is straightforward that~\ref{itm:1} is a consequence of the definition of time progress in the planning semantics. For Relation~\ref{itm:2},  
using Definition~\ref{def:plan}, we can deduce that:\\
\begin{displaymath}
(\loc,\val,\pi)\tranbp{\alpha}{3}(\loc',\val',\pi')\Rightarrow \pi(\alpha)=0,
\end{displaymath}
By Theorem~\ref{thm:pi_pln}, this implies that $\plnIntxt{\alpha}{0}$ is $\true$ at state $(\loc,\val,\pi)$, meaning that $\enabled(\alpha)$ is
also $\true$, which allows to infer~\ref{itm:2}.  

\begin{corollary}\label{cr:reach}
  If a state $(\loc,\val,\pi)\in\reach(S_p)$, then $(\loc,\val)\in\reach(S_g)$.
\end{corollary}

\begin{definition}[Weak Simulation]
  A weak simulation over $A=(\Q_A,\sum\cup\{\beta\},\to_A)$ and $B=(\Q_B,\sum\cup\{\beta\},
  \to_B)$ is a relation $R\subseteq \Q_A\times \Q_B$ such that we have: 
  $\forall(q,r)\in R, a\in \sum .q\transit{a}_A q' \implies\exists r':(q',r')\in R\wedge r\transit{
  \beta^*a\beta^*}_B r' \text{ and } \forall(q,r)\in R: q\transit{\beta}_Aq'\implies\exists r':
  (q',r')\in R\wedge r\transit{\beta^*}r'$.
  B simulates A, denoted by $A\simu{R}B$, means that B can do everything A does.
\end{definition}
The definition of weak simulation is based on the unobservability of $\beta-$transitions. In our case, $\beta-$transitions corresponds to $\plana-$transitions.
\begin{corollary}\label{cr:sim}
  $S_p\simu{R_1} S_g$ with $R_1=\{(\q,\pi);\q)\in\Q_p\times\Q_g\}$.
\end{corollary}

Corollary~\ref{cr:sim} corresponds to a notion of correctness of the planning semantics: any execution in the planning semantics corresponds to an execution in the global state semantics.

By contrast to (cite fm), where planning immediately, i.e. with a zero horizon, was allowed, 
by introducing a minimal horizon of planning interactions, the planning semantics does no longer
preserve all execution sequences of the global state semantics.

As explained in example, the planning semantics may introduce deadlocks as shown by the scenario presented in Example~\ref{exp:dl}.
In the following, we present different verification methods to check the deadlock freedom of a system under the planning semantics rules.





