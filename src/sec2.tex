\section{Timed Systems and Properties}
\label{sec2}

In the framework of the present paper, components are timed automata
and systems are compositions of timed automata with respect to
multiparty interactions. The timed automata we use are essentially
the ones from~\cite{alur94}, however, slightly adapted to embrace a
uniform notation throughout the paper.

\begin{definition}[Component]
  \label{def:atom}
  A \emph{component} is a tuple
  \\$\ta$ where
  $\Loc$ is a finite set of \emph{locations}, $\loc_0 \in \Loc$ is an \emph{initial} location, $\A$
  a finite set of \emph{actions}, $\X$ is a finite set of
  \emph{clocks}, $\T \subseteq \Loc\times(\A\times \tcal{C}
  \times 2^{\X})\times \Loc$ is a set of \emph{transitions} labeled with an action, a
  guard, and a set of clocks to be reset,
  and $\inv: \Loc\rightarrow\tcal{C}$ assigns a \emph{time
  progress condition}, $\tpc{\loc}$,
    to each location, where $\tcal{C}$ is the set of \emph{clock
  constraints} defined by the following grammar: 
  \begin{displaymath} 
    \tcal{C} := \true \mid x\sim\mathit{ct} \mid x - y \sim\mathit{ct} \mid \tcal{C} 
    \wedge \tcal{C}\mid\false, 
\end{displaymath}
  with $x, y\in \X$, $\sim$ $\in\{<,\le,=,\ge,>\}$ and $\mathit{ct} \in
  \mathbb{R}_{\ge 0}$.
 Time progress conditions are restricted to conjunctions of
  constraints of the form $x \leq \mathit{ct}$.
\end{definition}

Throughout the paper, we consider that components are deterministic timed automata, that is, 
at a given location $\loc$ and for a given action $a$, there is at most one outgoing transition from $\loc$ labeled by $a$. 
Given a timed automaton $\ta$, we write $\loc \transit{a,g,r} \loc'$ 
if there exists a transition $\tau = \big(\loc,(a,g,r),\loc'\big) \in\T$. 
We also write: 
\begin{displaymath}
  \guard{a}{\loc}=
  \begin{cases}
    g, &\text{ if } \exists\tau = \big(\loc,(a,g,r),\loc'\big) \in\T\\
    \false, &\text{ otherwise }
  \end{cases}
\end{displaymath}
Let $\Val$ be the set of all clock valuation functions
$\val: \X\rightarrow \mathbb{R}_{\ge
  0}$. 
For a clock constraint $C$, $C(\val)$ is a boolean value corresponding to the evaluation of $C$ on
$\val$. For a valuation $v \in \Val$, $\val + \delta$ is the valuation satisfying
$(\val + \delta)(x) = \val(x) + \delta$, while for a subset of clocks $r$, $\val[r]$ 
is the valuation obtained from $\val$ by resetting clocks of $r$, i.e. 
$\val[r](x) = 0$ for $x\in r$, $\val[r](x) = \val(x)$ otherwise.
We also denote by $C+\delta$ the clock constraint $C$ shifted by $\delta$, i.e. such that $C(\val+\delta)$ iff $C(\val)$.
\begin{definition}[Semantics]
  \label{def:atom_sem}
  A component $B = \ta$ defines the labeled transition system (LTS) 
  $\lts$ where $\Q\subseteq\Loc\times\Val(\X)$ denotes the \emph{states} of $B$, 
  and  $\rightarrow\subseteq\Q\times(A\cup\realpos) \times
  Q$ denotes the set of transitions between states according to the rules:
  \begin{itemize}[leftmargin=4mm]
    \item $(\loc,\val) \transit{a} (\loc',\val[r])$ if $\loc \transit{a,g,r} \loc'$, and $g(\val)$ is true (action step).
      \item $(\loc,\val) \transit{\delta} (\loc,\val+\delta)$ if $\tpc{\loc}(\val+\delta)$ for $\delta\in\realpos$
      (time step).
    \end{itemize}
\end{definition}
We define the predicate $\urg(\tpc{\loc})$ characterizing the urgency of a time progress condition
$\tpc{\loc}=\bigwedge\limits_{i=1}^m x_i\le ct_i$ at a state $(\loc,\val)$ as follows:
\begin{displaymath}
\urg(\tpc{\loc})=\bigvee_{i=1}^m (x_i=ct_i),
\end{displaymath}
An \emph{execution sequence} of $B$ from a state $(\loc,\val)$ is a path in the LTS starting at $(\loc,\val)$ and that 
alternates action steps and time steps, that is:
\begin{displaymath}
  (\loc_1,\val_1) \transit{\ex_1} \ldots \transit{\ex_i} (\loc_n,\val_n),n\in\integerpos,\sigma\in \A\cup\realpos
\end{displaymath}
In this paper, we always assume components with \emph{well formed guards} meaning that transitions $\loc \transit{a,g,r} \loc'$ satisfy
$g(v)\Rightarrow\tpc{\loc}(\val)\wedge\tpc{\loc'}(\val[r])$ for any $v \in \Val$.
%Before defining the semantics of a component, we first fix some
%notation.
We say that a state $(\loc,\val)$ is \emph{reachable} if there is an execution sequence from the initial configuration $(\loc_0, \val_0)$ leading to 
$(\loc,\val)$, where $\val_0$ assigns $0$ to all clocks.
Notice that the set of reachable states is in general infinite, but it can be partitioned into a finite number of symbolic
states~\cite{tripakis98:thesis,bengtssonY03,henzinger94}.
A symbolic state is defined by a pair $(\loc,\zeta)$ where, 
$\loc$ is a location of $B$, and $\zeta$ is a zone, i.e. a set of clock valuations defined by a clock constraint (as defined in Definition~\ref{def:atom}).
Efficient algorithms for computing symbolic states and operations on zones are fully described in~\cite{bengtssonY03}.
Given symbolic states $\{ (\loc_j,\zeta_j) \}_{j \in J}$ of $B$, the predicate $\reach(B)$ characterizing the reachable states can be formulated as:
\begin{displaymath}
 \reach(B) = \bigvee_{j \in J} \at{\loc_j} \wedge \zeta_j, 
\end{displaymath}
where $\at{\loc_j}$ is true on states whose location is $\loc_j$, and clock constraint $\zeta_j$ is straightforwardly applied to clock valuation functions of states.

We also define the predicate $\enabled(a)$ characterizing states $(\loc, \val)$ at which an action $a$ is enabled, i.e. such that $(\loc,\val) \transit{a} (\loc',\val')$. It can be written:
\begin{displaymath}
\enabled(a) = \bigvee_{(\loc,a,g,r,\loc') \in T} \at{\loc}  \wedge \guard{a}{\loc}
\end{displaymath}


\begin{definition}[Deadlock]
  \label{def:dl}
  We say that a state $(\loc,\val)$ of a component B deadlocks,  
  if neither action steps nor time steps can be done from this state. The following equation characterizes those states:
  \vspace{-1mm}
  \begin{displaymath}
   \forall a\in\A. \ \neg\enabled(a) \ \wedge \ \urg(\tpc{\loc})
  \end{displaymath}
\end{definition}


  In our framework, components communicate by means of
  \emph{multiparty interactions}.
  A multiparty interaction is a rendez-vous synchronization between actions of
  a fixed subset of components. It takes
  place only if all the participants agree to execute the corresponding actions.
  Given $n$ components $B_i$, $i=1,\dots,n$, with disjoint sets of actions
  $\A_i$, an interaction is a subset of actions $\alpha \subseteq \cup_{1 \leq  i \leq n}\A_i$
  containing at most one action per component, i.e. $\alpha \cap\A_i$ is either empty or a singleton $\{ a_i \}$.
  That is, an interaction 
  $\alpha$ can be put in the form $\{ a_i \}_{i \in I}$ with $I \subseteq \{ 1, \ldots, n \}$ and $a_i \in\A_i$ for all $i \in I$.
%\begin{definition}[Composition] 
%  \label{def:comp}
%  For $n$ components $B_i = (\Loc_i,
%  \loc^i_0, A_i, T_i, \X_i, \inv_i)$, with $\Loc_j \cap \Loc_j = \emptyset$, $A_i \cap A_j = 
%  \emptyset$, and $\X_i \cap \X_j = \emptyset$ for any $i \neq j$, the
%  \emph{composition} $\gamma(B_1, \ldots, B_n)$ w.r.t. a set of interactions
%  $\gamma$ is defined by a timed automaton
%  $S = (\Loc,\loc_0,\gamma,T_{\gamma},\X,\inv)$ where $\loc_0 =
%  (\loc^1_0,\dots, \loc^n_0)$, $\X=\X_1\cup\ldots\cup\X_n$, $\Loc=\Loc_1\times\ldots\times\Loc_n$,
%  $\inv=\inv_1 \wedge \ldots \wedge \inv_n$ for $\loc=(\loc_1,\dots,
%  \loc_n)$, and $T_{\gamma}$ is such that $\loc\xrightarrow{\alpha,g,r}{}\loc'$ for 
%  $\alpha=\{a_i\}_{i\in I}$, $\loc=(\loc_1,\dots,\loc_n)$, and $\loc'=(\loc'_1,\dots
%  ,\loc'_n)$, if for $i\not\in I$ we have $\loc'_i = \loc_i$, and for $i\in I$ we have 
%  $\loc_i\xrightarrow{a_i,g_i,r_i}\loc'_i$, and $g_{\alpha}=\bigwedge_{i\in I}g_i$ and $r=\bigcup_{i\in I}r_i$.
%\end{definition}

In practice we do not explicitly build compositions of components.
We rather interpret their semantics at runtime by evaluating enabled interactions based on current states of components.
In a composition of $n$ components $B_{i\in\{1,\cdots,n\}}$, denoted by $\gamma(B_1, \ldots, B_n)$,
an action $a_i$ can execute only as part of an interaction $\alpha$ such that $a_i \in
\alpha$, that is, along with the execution of all other actions $a_j
\in \alpha$, which corresponds to the usual notion of multiparty
interaction.

\begin{definition}[Semantics of a Composition]
  \label{pt:comp_sem}
  Given a set of components $\{B_1,\cdots,B_n\}$ and an interaction set $\gamma$.
  The semantics of the composition $S =\gamma(B_1,\cdots,B_n)$ w.r.t the set of interactions 
  $\gamma$, is the LTS $(\Q_g,\gamma\cup\realpos,\to_{\gamma})$ where:
\begin{itemize}[leftmargin=5mm]
  \item $\Q_g=\Loc\times\mathcal{V}(\X)$ is the set of global states, where $\Loc=\Loc_1\times
  \cdots\times\Loc_n$ and $\X=\bigcup^n_{i=1}\X_i$. We write a state $\q=(\loc,\val)$
  where $\loc=(\loc_1,\cdots,\loc_n)\in\Loc$ is a global location and $\val=(\val_1,
  \cdots,\val_n)\in\Val(\X)$ is global clocks valuations.
  \item $\to_{\gamma}$ is the set of labeled transitions defined by the rules: 
  \begin{itemize}
    \item $(\loc,\val)\transit{\alpha}_{\gamma}(\loc',\val')$ if for $\alpha=\{a_i\}_{i\in I}\in\gamma.
      \forall i\in I.(\loc_i,\val_i)\transit{a_i}(\loc_i,\val_i)$ and $\forall i \notin I.(\loc_i,\val_i)=(\loc_i',\val'_i)$
      (Interaction step).
        \item $(\loc,\val)\transit{\delta}_{\gamma}(\loc,\val+\delta)$ if
          $\forall i\in\{1,\cdots,n\}\quad \inv_{\loc_i}(\val_i+\delta)$ for $\delta\in\realpos$ (time step).
      \end{itemize}
  \end{itemize}
\end{definition}

%In what follows, we consider only deadlock-free systems w.r.t the presented semantics.
By abuse of notation, predicates $\at{\loc_i}$ of individual components $B_i$ are interpreted on states of $S$,
being true for $(\loc,\val)$ iff $B_i$ is at location $\loc_i$ in $\loc$, i.e. iff $\loc\in
\Loc_1\times\ldots\times\Loc_{i-1}\times\{\loc_i\}\times\Loc_{i+1}\times\ldots\times\Loc_n$.
Similarly, clock constraints of components $B_i$ are applied to clock valuation functions $\val$ 
of the composition by restricting $\val$ to clocks $\X_i$ of $B_i$.
Given an interaction $\alpha\in\gamma$, these notations allow us to write $\enabled(\alpha)$ as:
\begin{align*}
  \enabled(\alpha) &= \bigvee_{\loc=(\loc_1,\cdots,\loc_n)\in\Loc_{\alpha}} \at{\loc}\wedge \guard{\alpha}{\loc},\\
                   &= \bigvee_{(\loc_1,\cdots,\loc_n)\in\Loc_{\alpha}} \at{\loc}\wedge \bigwedge_{a_i\in\alpha}\guard{a_i}{\loc_i},\\
                   &= \bigvee_{(\loc_1,\cdots,\loc_n)\in\Loc_{\alpha}} \Big(\bigwedge_{i=1}^n\at{\loc_i}\wedge \bigwedge_{a_i\in\alpha}\guard{a_i}{\loc_i}\Big),\\
                   &= \bigwedge_{a_i\in\alpha} \enabled(a_i),
\end{align*}
where $\Loc_{\alpha}=\{\loc\in\Loc|\loc\transit{\alpha,g,r}\loc'\}.$


\begin{figure*}[t]
 \centering
\begin{tikzpicture}[->,node distance=1.3cm,>=stealth',bend angle=20,auto,
  place/.style={circle,thick,draw=blue!75,fill=blue!20,minimum size=10mm},
  red place/.style={place,draw=red!75,fill=red!20}
  every label/.style={red},
  every node/.style={scale=.6},
  dots/.style={fill=black,circle,inner sep=2pt},
  initial text={}]

  \node [accepting, place] (l0)  {$\loc_0^1$};
  \node [place,below=1.5cm of l0,label={[shift={(-2,-1.9)}]$C$}] (l1) {$\loc_1^1$};

  \path (l0) edge [in=100, out=160,loop,looseness=4] node[left]{$end_0$} (l0)
         edge [bend left] node[right,align=center]{$init_0$\\$z>4$} (l1)
    (l1) edge [in=-10, out=-70,loop,looseness=4] node[right]{$end_0$} (l1)
         edge [bend left] node[left,align=center]{$run$\\$z:=0$} (l0);

  \node [accepting, place] (p1-0) [right=2cm of l0] {$\loc_0^2$};
  \node [place] (p1-1) [right=1.5cm of p1-0]{$\loc_1^2$};
  \node [place] (p1-2) [below=1.5cm of p1-1,label=right:\textcolor{red}{$x\le 3$},label={[shift={(1,-1.9)}]$T_1$}]            {$\loc_2^2$};
  \node [place] (p1-3) [left=1.5cm of p1-2] {$\loc_3^2$};

  \path (p1-0) edge node[align=center, pos=0.5]{$init_1$} (p1-1)
    (p1-1) edge node[align=center, pos=0.5]{$start_1$\\$x:=0$ } (p1-2)
    (p1-2) edge node[align=center, pos=0.5]{$process_1$ \\$1\le x\le3$ } (p1-3)
    (p1-3) edge node[align=center, pos=0.5]{$end_1$} (p1-0);


  \node [place] (p2-1) [left=2cm of l0]{$\loc_1^3$};
  \node [accepting, place] (p2-0) [left=1.5cm of p2-1]{$\loc_0^3$};
  \node [place] (p2-2) [below=1.5cm of p2-1,label=right:\textcolor{red}{$y\le 3$},label={[shift={(-4.8,-1.9)}]$T_2$}]            {$\loc_2^3$};
  \node [place] (p2-3) [left=1.5cm of p2-2] {$\loc_3^3$};

  \path (p2-0) edge node[align=center, pos=0.5]{$init_2$} (p2-1)
    (p2-1) edge node[align=center, pos=0.5]{$start_2$\\ $y:=0$ } (p2-2)
    (p2-2) edge node[align=center, pos=0.5]{$process_2$\\$1\le y\le3$ } (p2-3)
    (p2-3) edge node[align=center, pos=0.5]{$end_2$} (p2-0);

  \node [accepting, place] (r0) [above=2cm of l0,xshift=-2.25cm,label={[shift={(-1,-2)}]$R$}] {$\loc_0^4$};
  \node [place,right=2cm of r0] (r1) {$\loc_1^4$};

  \path (r0) edge [bend left] node[above]{take} (r1)
    (r1) edge [bend left] node[below]{free} (r0);

  \node [inner xsep=2cm,inner ysep=2cm,draw, yshift=-1mm, fit=(l0)(l1)] (rec1) {};
  \node [inner xsep=2cm,inner ysep=1.5cm,draw, fit=(r0)(r1)] (rec2) {};
  \node [inner xsep=2cm,inner ysep=2cm,draw, fit=(p1-0)(p1-1)(p1-2)(p1-3)] (rec3) {};
  \node [inner xsep=2cm,inner ysep=2cm,draw, fit=(p2-0)(p2-1)(p2-2)(p2-3)] (rec4) {};
%  \node [inner xsep=4cm,inner ysep=2.5cm,draw, fit=(rec1)(rec2)(rec3)(rec4)] (rec5) {};
 % \node [inner xsep=1.5cm,inner ysep=5mm,draw,above=5mm of rec1] (rec5) {

  %  $\begin{aligned}
  %    \gamma &=\{
  %  init_1=\{init,init_1\}, start_1=\{start,start_1\}, process_1=\{enter,proces_1\}, 
  %  end_1=\{end,exit,end_1\}, \\
   %  & init_2=\{init, init_2\}, start_2=\{start,start_2\}, 
   % process_2=\{enter,process_2\},end_2=\{end,exit,end_2\}\} 
  %  \end{aligned}$
 % };

  \node [dots,label=90:$init_2$] (i2) at ($(rec4.south west)!0.6!(rec4.south east)$) {};
  \node [dots,label=90:$start_2$] (s2) at ($(rec4.south west)!0.8!(rec4.south east)$) {};
  \node [dots,label=-90:$end_2$] (e2) at ($(rec4.north east)!0.2!(rec4.north west)$) {};
  \node [dots,label=-90:$process_2$] (p2) at ($(rec4.north east)!0.5!(rec4.north west)$) {};

  \node [dots,swap,label=90:$init_1$] (i1) at ($(rec3.south east)!0.6!(rec3.south west)$) {};
  \node [dots,swap,label=90:$start_1$] (s1) at ($(rec3.south east)!0.8!(rec3.south west)$) {};
  \node [dots,swap,label=-90:$end_1$] (e1) at ($(rec3.north west)!0.2!(rec3.north east)$) {};
  \node [dots,swap,label=-90:$process_1$] (p1) at ($(rec3.north west)!0.5!(rec3.north east)$) {};

  \node [dots,label=-90:take] (tr) at ($(rec2.north west)!0.5!(rec2.north east)$) {};
  \node [dots,label=90:free] (fr) at ($(rec2.south west)!0.5!(rec2.south east)$) {};

  \node [dots,label=90:$init_0$] (ic) at ($(rec1.south west)!0.4!(rec1.south east)$) {};
  \node [dots,label=90:$run$] (rc) at ($(rec1.south west)!0.6!(rec1.south east)$) {};
  \node [dots,label=-90:$end_0$] (ec) at ($(rec1.north west)!0.5!(rec1.north east)$) {};
  
  \path (tr) ++(0,0.25cm) +(-1cm,0) coordinate(xp2) +(1cm,0) coordinate(xp1);
  \draw  [-] (p1) |-node[above,xshift=-2.5cm]{$\alpha_5$} (xp1) -- (tr) -- (xp2)node[above,xshift=-2.5cm]{$\alpha_6$} -| (p2);

  \path (ic) ++(0,-0.5cm) +(1cm,0) coordinate(xi1) +(-1cm,0) coordinate(xi2);
  \draw[-,name path=line1] (i1) |-node[above,xshift=-2.5cm]{$\alpha_1$} (xi1) -- (ic) -- (xi2)node[above,xshift=-2.5cm]{$\alpha_2$} -| (i2); %here


  \path (rc) ++(0,-1cm) +(1cm,0) coordinate(sx1) +(-1cm,0) coordinate(sx2);

  \path[-,name path=line2] (s1) |-node[above,xshift=-1.5cm]{$\alpha_3$} (sx1) -- (rc) -- (sx2)node[above,xshift=-2cm]{$\alpha_4$} -| (s2); %here

 \path[name intersections={of=line1 and line2, by={a,b,c,d}}];% here

% draw semicircles at crossing points on the path
\coordinate (aux1) at (s2|-sx2);
\coordinate (aux2) at (s1|-sx1);
\draw[-,connect=(s2) to (aux1) over (d) by 3pt];
\draw[-] (aux1) -- (sx2);
\draw[-,connect=(sx2) to (rc) over (c) by 3pt];
\draw[-,connect=(rc) to (sx1) over (b) by 3pt];
\draw[-] (sx1) -- (aux2);
\draw[-,connect=(aux2) to (s1) over (a) by 3pt];

\path (fr) ++(0,-2.5mm) +(-1cm,0) coordinate(xe2) +(1cm,0) coordinate(xe1);
  \draw [-] (e1) |- node[above,xshift=-2cm]{$\alpha_7$}(xe1) -- (ec);
  \draw [-] (xe1) -- (fr);
  \draw [-] (e2) |- node[above,xshift=2cm]{$\alpha_8$}(xe2) -- (ec);
  \draw [-] (xe2) -- (fr);
% \path (i1) ++(-0.5cm,0) coordinate(xi1);
 % \path (i2) ++(0.2cm,0) coordinate(xi2);
 % \path (s1) ++(-0.2cm,0) coordinate(xs1);
 % \path (s2) ++(0.2cm,0) coordinate(xs2);
 % \path (e1) ++(-0.3cm,0) coordinate(xe1);
 % \path (e2) ++(0.3cm,0) coordinate(xe2);
 % \path (p1) ++(-0.2cm,0) coordinate(xp1);
 % \path (p2) ++(0.2cm,0) coordinate(xp2);
 % \draw  [-] (i1) -- (xi1) -- (ic);
 % \draw  [-] (i2) -- (xi2) -- (ic);
 % \draw  [-] (s1) -- (xs1) -- (rc);
 % \draw  [-] (s2) -- (xs2) -- (rc);
 % \draw  [-] (e2) -- (xe2) -- (ec);
 % %\draw  [-] (xe2) -- (fr);
 % \draw  [-] (e1) -- (xe1) -- (ec);
%  \draw  [-] (xe1) -- (fr);
\end{tikzpicture}
  \caption{Task Manager}
 \label{fig:run}
\end{figure*}  




\begin{example}[Running Example]
  \label{exp:run}
  Let us consider as a running example the composition of four components $C$, $T_1$, $T_2$, and $R$ of Figure~\ref{fig:run}.
  Component $C$ represents a  controller that initializes, releases, and ends tasks $T_1$ and $T_2$.
  Tasks use the shared resource $R$ during their execution.
  To implement such behavior, we consider the following interactions between $C$, $R$, and $T_1$: $\alpha_1=\{init_0, init_1\}$,
    $\alpha_3=\{start_0, start_1\}$, $\alpha_5=\{ take, process_1\}$, $\alpha_7 = \{end_0, free, end_1 \}$, 
      and similar interactions $\alpha_2$, $\alpha_4$, $\alpha_6$, $\alpha_8$ for task $T_2$, 
      as shown by connections on Figure~\ref{fig:run}.
      The controller is responsible for firing
      the execution of each task. First, it non-deterministically initializes one
      of the two tasks, i.e. executes $\alpha_1$ or $\alpha_2$, and then
      releases it through interaction $\alpha_3$ or $\alpha_4$.
      Tasks perform their processing independently of the controller, after being granted an access to the shared resource ($\alpha_5$ or $\alpha_6$).
      When ended by the controller, a task releases the resource  (interactions $\alpha_7$ or $\alpha_8$) and go back to its initial location.
      An example of execution sequence of the system of Figure~\ref{fig:run} is given below, in which valuations $v$ of clocks $x$, $y$, and $z$ are represented as a tuples $(\val(x),\val(y),\val(z))$:
      \begin{displaymath}
      %  \scriptsize{
        \small{
        \begin{split}
          &((\loc_0^1,\loc_0^2,\loc_0^3,\loc_0^4),(0,0,0))\transit{5}_{\gamma}((\loc_0^1,\loc_0^2,\loc_0^3,\loc_0^4),(5,5,5))\transit{\alpha_1}_{\gamma}\\
          &((\loc_1^1,\loc_1^2,\loc_0^3,\loc_0^4),(5,5,5))\transit{\alpha_3}_{\gamma}((\loc_0^1,\loc_2^2,\loc_0^3,\loc_0^4),(0,5,0))\transit{2}_{\gamma}\\
          &((\loc_0^1,\loc_2^2,\loc_0^3,\loc_0^4),(2,7,2))\transit{\alpha_5}_{\gamma}((\loc_0^1,\loc_3^2,\loc_0^3,\loc_1^4),(2,7,2))\transit{3}_{\gamma}\\
          &((\loc_0^1,\loc_3^2,\loc_0^3,\loc_1^4),(5,10,5))\transit{\alpha_2}_{\gamma}((\loc_1^1,\loc_3^2,\loc_1^3,\loc_1^4),(5,10,5))\\
        \end{split}
      }
      \end{displaymath}
    
\end{example}
