\section{Appendix}
\begin{proof}[Theorem~\ref{thm:dl}]

Let $(\loc,\val,\pi)$ be a reachable state of the planning semantics.
Assuming that $(\loc,\val,\pi)$ is a deadlock state, this means that 
\emph{(i)} time cannot progress in the system, 
\emph{(ii)} no interactions can be planned or \emph{(iii)} executed.

By definition of the planning semantics we have:\\
\begin{description}[labelwidth=1.5cm]
  \setlength\itemsep{1em}
  \item[\namedlabel{p1}{R1}] $\emph(i)\Rightarrow
    \bigvee_{B_i\in S\setminus\p{\pi}}\bigvee_{\loc_i\in\Loc_i}\at{\loc_i}
    \wedge\urg(\tpc{\loc_i}(\val_i+h_{\min}))$
  \item[\namedlabel{p2}{R2}] $\emph(iii)\Rightarrow\forall\alpha\in\gamma\setminus\pi,
    \neg\plntxt{\alpha}\vee\alpha\in\confl(\pi)$
  \item[\namedlabel{p3}{R3}] $\emph(ii)\Rightarrow\forall\alpha\in\pi,\pi(\alpha)\neq0
        \wedge\plnIntxt{\alpha}{\pi(\alpha)}$
\end{description}

From~\ref{p1} and~\ref{p2} we can deduce that interactions 
involving the component disallowing the time progress cannot be planned either because
they are conflicting with the plan or not plannable, i.e., not in their respective
planning states. This allows to write:
\begin{adjustwidth}{-8mm}{-8mm}
\begin{equation}\label{eq:pr1}
  \bigvee_{B_i\in S\setminus\p{\pi}}\Big[\Big(\bigvee_{\loc_i\in\Loc_i}\at{\loc_i}
    \wedge\urg(\tpc{\loc_i}(\val_i+h_{\min}))\Big)\wedge
    \Big(\bigwedge_{\alpha\in\Gamma(B_i)\setminus\confl(\pi)}\hspace{-8mm}
    \neg\plntxt{\alpha}\vee\hspace{-5mm}
    \bigvee_{\alpha\in\Gamma(B_i)\cap\confl(\pi)}\hspace{-8mm}\plntxt{\alpha}\Big)\Big]
\end{equation}
\end{adjustwidth}

By combining Equation~\ref{eq:pr1} and~\ref{p2}, we obtain Equation~\ref{eq:dl} of 
Theorem~\ref{thm:dl}.
\end{proof}
\qed


\begin{proof}[Theorem~\ref{thm:dla}]
  Let $(\loc,\val,\pi)$ be a reachable deadlock state of the planning semantics. This 
  means that Equation~\ref{eq:dl} is satisfied, meaning that $\exists\alpha\in\pi$ 
  satisfying:
  \begin{equation}\label{1}
    \Big(\bigwedge_{\beta\in\Gamma(B_i)\setminus\confl(\alpha)}\hspace{-8mm}
    \neg\plntxt{\beta}\vee\hspace{-5mm}\bigvee_{\beta\in\Gamma(B_i)\cap\confl(\alpha)}
    \hspace{-8mm}\plntxt{\beta}\Big)\wedge\pi(\alpha)\neq0\wedge
    \plnIntxt{\alpha}{\pi(\alpha)}
  \end{equation}
  We have:
  \begin{align}
  \pi(\alpha)\neq0\wedge\plnIntxt{\alpha}{\pi(\alpha)}
  &\Rightarrow\tilde{\plntxt{\alpha}}\label{2}\\
  \bigvee_{B_i\in S\setminus\p{\pi}}\bigvee_{\loc_i\in\Loc_i}\at{\loc_i}\wedge\urg(\tpc{\loc_i}(\val+h_{\min}))&\Rightarrow\bigvee_{B_i\in S\setminus\p{\alpha}}\bigvee_{\loc_i\in\Loc_i}\at{\loc_i}\wedge\urg(\tpc{\loc_i}(\val+h_{\min})\label{3}
   \end{align}
     
   By combining~\ref{1},~\ref{2} and~\ref{3} we obtain Equation~\ref{eq:dla}, which
   proves Theorem~\ref{thm:dla}


\end{proof}
\qed
