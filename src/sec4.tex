\section{Planning Semantics Correctness}
\label{sec4}
Local planning of interactions consists effectively in applying a local time step, followed
by the execution of an interaction. Even if local time steps are allowed in planned components,
it may be disallowed in the rest of the system. Additionally, planning horizons 
insert a certain latency between interactions, which can result in missing interactions deadlines in some cases.
Thus, as explained in Example, the planning semantics may 
exhibit deadlock situations nonexistent in the system under the global state semantics, since:
\emph{(i)} it does not consider time progress conditions of components not participating in 
planned interactions, and \emph{(ii)} it introduces a certain latency between interaction executions. 

Since the application model does not consider delays between interactions,
in case of urgency of a time progress condition, semantically, consecutive executions can happen
in the global state semantics and enable thus to get ride of this urgency. Thing that does not hold in the planning semantics.
In what follows, we present different methodologies in order to enforce system correctness under
the planning semantics rules. 

\subsubsection{Planning Semantics Restriction}\mbox{}\\
\label{sec4:1}
For the sake of simplicity, we assume that all interactions have the same lower bound  
planning horizon, that is, $\forall\alpha,\beta\in\gamma$ such that $\alpha\neq\beta$,
$\hmn=\hmnb{\beta}=h_{\min}$.
In order to enforce the correctness of the planning semantics, we restrict the condition
of time progress as follows:
\begin{equation}
\label{eq:tpc_rest}
\begin{split}
  &\delta\le \text{next } \pi
  \wedge\forall B_i\notin\p{\pi}.\tpc{i}(\val_i+\delta+h_{\min})\\
  %\urg(\tpc{\loc_i}(\val_i+\delta+\delta'))\forall\delta'<h_{\min}\\
        \cline{1-2}
        &\hspace{12mm}(\loc,\val,\pi)\tranbp{\delta}{3}(\loc,\val+\delta,\pi-\delta)\\
\end{split}
\end{equation}
This restriction aims to disallow any time progress, if there is a component not
participating in any interaction of the plan such that it progresses out of its
allowed planning intervals.

Let $\dot{S_{p}}=(\Q_{p},\gamma\cup\realpos\cup\{\plana\},\tranbp{}{3})$
the labeled transition system characterizing the planning semantics under
the restriction~\ref{eq:tpc_rest}. 

\begin{corollary}\label{cr:sim2}
  $\dot{S_p}\simu{R_1} S_g$ with $R_1=\{(\q,\pi);\q)\in\Q_p\times\Q_g\}$.
\end{corollary}

\subsubsection{Deadlock Characterization}

%\begin{theorem}[Local Deadlock]
%  \label{thm:local_dl}
%  Let $\{B_1,\cdots,B_n\}$ be a set of components and $\gamma$ an interaction set. 
%  Let $\loc_i\in\Loc_i$ be a location of a component $B_{i\in\{1,\cdots,n\}}$ 
%  such that $\tpc{\loc_i}\neq\true$. Let $\Gamma(\loc_i)$ be the subset of interactions
%  including all actions possible from the location $\loc_i$, that is,
%  $\Gamma(\loc_i)=\{\alpha|\exists\loc_i\transit{a_i,g_i,r_i}\loc'_i.a_i\in\alpha\}$.
%  If it exists a reachable state $(\loc,\val,\pi)$ of the planing semantics 
%  such that: 
%  \[\exists\loc_i\in\loc,\val_i=\tpc{\loc_i-h_{\min}}.\forall\alpha\in\Gamma\ \neg\plntxt{\alpha}\] 
%  then component the $B_i$ locally deadlocks under the planning semantics rules, and thus, the composition under 
%  the planning semantics rules deadlocks.
%\end{theorem}

%Condition of Theorem~\ref{thm:local_dl} enables to identifies models where the time progress condition
%of a component is affected by the execution of interactions including other components.

\begin{theorem}
  \label{thm:dl}
  If a reachable state $(\loc,\val,\pi)$ of the composition $S=\gamma(B_1,\cdots,B_n)$ under
  the planning semantics rules deadlocks, then the following is satisfied:
  \begin{adjustwidth}{-10mm}{-10mm}
  \begin{equation}\label{eq:dl}
  \begin{split}
    &\bigvee_{B_i\in S\setminus\p{\pi}}\Big[\Big(
    \underbrace{\bigvee_{\loc_i\in\Loc_i}\at{\loc_i}
  \wedge\urg(\tpc{\loc_i}(\val_i+h_{\min}))\Big)}_\text{\mytag{A}{A}}\wedge
    \Big(\underbrace{\bigwedge_{\alpha\in\Gamma(B_i)\setminus\confl(\pi)}\hspace{-8mm}
    \neg\plntxt{\alpha}\vee\hspace{-5mm}\bigvee_{\alpha\in\Gamma(B_i)\cap\confl(\pi)}
\hspace{-8mm}\plntxt{\alpha}}_\text{\mytag{B}{B}}\Big)\Big]\\
&\hspace{4.5cm}\wedge\underbrace{\bigwedge_{\alpha\in\pi}\Big(\pi(\alpha)\neq0\wedge
    \plnIntxt{\alpha}{\pi(\alpha)}\Big)}_\text{\mytag{C}{C}} 
  \end{split}
  \end{equation}
  \end{adjustwidth}
\end{theorem}

Theorem~\ref{thm:dl} characterizes deadlock states of the planning semantics. Precisely,
terms A of Equation~\ref{eq:dl} represents the time progress restriction as presented
in~\ref{eq:tpc_rest}. It expresses clocks valuations, in a component not participating 
in the planned interaction, $h_{\min}$ units of time before its expiration. On the other hand,
term B describes the cause of the time progress restriction, that is, all interactions
of $\Gamma(\loc_i)$ are not plannable either because (1) they are conflicting with the plan
or, (2) because they are not in their corresponding planning states.
Finally, term C depicts the existence of interactions in the plan and the fact that none of them
reached yet their execution dates.
 
\begin{theorem}\label{thm:dla}
  Let $\phi(\alpha)$ be the following predicate:

  \begin{adjustwidth}{-10mm}{-10mm}
    \begin{equation}\label{eq:dla}
  \begin{split}
    &\bigvee_{B_i\in S\setminus\p{\alpha}}\Big[\Big(
    \bigvee_{\loc_i\in\Loc_i}\at{\loc_i}
  \wedge\urg(\tpc{\loc_i}(\val_i+h_{\min}))\Big)\wedge
    \Big(\bigwedge_{\beta\in\Gamma(B_i)\setminus\confl(\alpha)}\hspace{-8mm}
    \neg\plntxt{\beta}\vee\hspace{-5mm}\bigvee_{\beta\in\Gamma(B_i)\cap\confl(\alpha)}
\hspace{-8mm}\plntxt{\beta}\Big)\Big]\\
&\hspace{6.5cm}\wedge\tilde{\plntxt{\alpha}} 
  \end{split}
\end{equation}
  \end{adjustwidth}
  where $\tilde{\plntxt{\alpha}}$ is the obtained predicate by replacing 
  $h_{\min}$ by 0 in $\plntxt{\alpha}$ and transforming the upper 
  timing constraints of interaction $\alpha$ to strict guards.
  If a reachable state $(\loc,\val,\pi)$ deadlocks then $\exists\alpha\in\gamma$ 
  such that $\phi(\alpha)$ is satisfied.
\end{theorem}

Since reachable states of the restricted planning semantics are reachable 
in the global state semantics (Corollary~\ref{cr:sim2}), we can deduce that if for
every interaction of $\gamma$, Equation~\ref{eq:dla} is unsatisfied, then the system
under the restricted planning semantics is deadlock-free.


