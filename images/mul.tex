\begin{figure}[H]
  \center
  \begin{adjustwidth}{-8mm}{-12mm}
\begin{tikzpicture}[->,node distance=1.3cm,>=stealth',bend angle=20,auto,
  place/.style={circle,thick,draw=blue!75,fill=blue!20,minimum size=8mm},
  red place/.style={place,draw=red!75,fill=red!20}
  every label/.style={red},
  every node/.style={scale=.7},
  dots/.style={fill=black,circle,inner sep=1pt},
  triangle/.style={fill=black,regular polygon,regular polygon sides=3,minimum size=10pt,inner sep=0pt,}
]
\tikzstyle{every node}=[font=\scriptsize]

  \node (rec1) at (0,0) [draw, thick,minimum width=1cm,minimum height=1cm]{$C$};
  \node (rec2) [draw, above=5mm of rec1,thick,minimum width=1cm,minimum height=1cm]{$R$};
  \node (rec3) [draw, right=1cm of rec1,thick,minimum width=1cm,minimum height=1cm]{$T_{1}$};
  \node (rec4) [draw, left=1cm of rec1,thick,minimum width=1cm,minimum height=1cm]{$T_{2}$};

  \node [dots] (i2) at ($(rec4.south west)!0.6!(rec4.south east)$) {};
  \node [dots] (s2) at ($(rec4.south west)!0.8!(rec4.south east)$) {};
  \node [dots] (e2) at ($(rec4.north east)!0.2!(rec4.north west)$) {};
  \node [dots] (p2) at ($(rec4.north east)!0.5!(rec4.north west)$) {};
  
  \node [dots] (i1) at ($(rec3.south east)!0.6!(rec3.south west)$) {};
  \node [dots] (s1) at ($(rec3.south east)!0.8!(rec3.south west)$) {};
  \node [dots] (e1) at ($(rec3.north west)!0.2!(rec3.north east)$) {};
  \node [dots] (p1) at ($(rec3.north west)!0.5!(rec3.north east)$) {};

  \node [dots] (tr) at ($(rec2.north west)!0.5!(rec2.north east)$) {};
  \node [dots] (fr) at ($(rec2.south west)!0.5!(rec2.south east)$) {};
  
  \node [dots] (ic) at ($(rec1.south west)!0.4!(rec1.south east)$) {};
  \node [dots] (rc) at ($(rec1.south west)!0.6!(rec1.south east)$) {};
  \node [dots] (ec) at ($(rec1.north west)!0.5!(rec1.north east)$) {};

  \path (tr) ++(0,0.5cm) +(-1cm,0) coordinate(xp2) +(1cm,0) coordinate(xp1);
  \draw  [-] (p1) |-node[above,xshift=-.5cm]{$process_1$} (xp1) -- (tr) -- (xp2)node[above,xshift=-.5cm]{$process_2$} -| (p2);

  \path (ic) ++(0,-0.5cm) +(3mm,0) coordinate(xi1) +(-3mm,0) coordinate(xi2);
  \draw  [-, name path=line1] (i1) |-node[above,xshift=-1cm]{$init_1$} (xi1) -- (ic) -- (xi2)node[above,xshift=-.5cm]{$init_2$} -| (i2);
  \path (rc) ++(0,-0.75cm) +(3mm,0) coordinate(sx1) +(-3mm,0) coordinate(sx2);
  \path  [-, name path=line2] (s1) |-node[below,xshift=-5mm]{$start_1$} (sx1) -- (rc) -- (sx2)node[below,xshift=-1cm]{$start_2$} -| (s2);
  
  \path[name intersections={of=line1 and line2, by={a,b,c,d}}];%
  
  \coordinate (aux1) at (s2|-sx2);
  \coordinate (aux2) at (s1|-sx1);
  \draw[-,connect=(s2) to (aux1) over (d) by 2pt];
  \draw[-] (aux1) -- (sx2);
  \draw[-,connect=(sx2) to (rc) over (c) by 2pt];
  \draw[-,connect=(rc) to (sx1) over (b) by 2pt];
  \draw[-] (sx1) -- (aux2);
  \draw[-,connect=(aux2) to (s1) over (a) by 2pt];

  \path (fr) ++(0,-0.225cm) +(-1cm,0) coordinate(xe2) +(1cm,0) coordinate(xe1);
  \draw [-] (e1) |- node[above,xshift=-.6cm]{$end_1$}(xe1) -- (ec);
  \draw [-] (xe1) -- (fr);
  \draw [-] (e2) |- node[above,xshift=.6cm]{$end_2$}(xe2) -- (ec);
  \draw [-] (xe2) -- (fr);


  \node (rec9) [draw, right=2cm of rec3,yshift=1.5cm,thick,minimum width=6cm,minimum height=1.5cm,align=center]{Scheduler};
  \path (rec9) ++ (2.5cm,3mm) coordinate (g);
  \path (g) ++ (1mm,1mm) coordinate (g1);
  \path (g) ++ (0,1.7mm) coordinate (g2) node [below=3mm of g]{$g$};
  \draw (g) circle [radius=2mm];
  \draw[-,line width=1pt] (g) -- (g1);
  \draw[-,line width=1pt] (g) -- (g2);
  \node (rec5) [draw, below=1cm of rec9,thick,xshift=-8mm,minimum width=1cm,minimum height=1cm]{$R^{SR}$};
  \node (rec6) [draw, right=7mm of rec5,thick,minimum width=1cm,minimum height=1cm]{$C^{SR}$};
  \node (rec7) [draw, left=7mm of rec5,thick,minimum width=1cm,minimum height=1cm]{$T_{1}^{SR}$};
  \node (rec8) [draw, right=7mm of rec6,thick,minimum width=1cm,minimum height=1cm]{$T_{2}^{SR}$};
 
  \node (p5) at ($(rec7.north west)!0.3!(rec7.north east)$)[yshift=-1mm] {};
  \node (q5) at ($(rec7.north west)!0.7!(rec7.north east)$)[yshift=-1mm] {};
  
  \node (p6) at ($(rec5.north west)!0.3!(rec5.north east)$)[yshift=-1mm] {};
  \node (q6) at ($(rec5.north west)!0.7!(rec5.north east)$)[yshift=-1mm] {};
  
  \node (p7) at ($(rec6.north west)!0.3!(rec6.north east)$)[yshift=-1mm] {};
  \node (q7) at ($(rec6.north west)!0.7!(rec6.north east)$)[yshift=-1mm] {};
  
  \node (p8) at ($(rec8.north west)!0.3!(rec8.north east)$)[yshift=-1mm] {};
  \node (q8) at ($(rec8.north west)!0.7!(rec8.north east)$)[yshift=-1mm] {};
   
  \node (ps5) [above=1cm of p5] {};
  \node (qs5) [above=1cm of q5] {};
  \node (ps6) [above=1cm of p6] {};
  \node (qs6) [above=1cm of q6] {};
  \node (ps7) [above=1cm of p7] {};
  \node (qs7) [above=1cm of q7] {};
  \node (ps8) [above=1cm of p8] {};
  \node (qs8) [above=1cm of q8] {};

  \draw [->] (p5) -- node[yshift=2mm]{Offer}(ps5);
  \draw [->] (qs5) -- node[yshift=-2mm]{Notif} (q5);
  \draw [->] (p6) -- node[yshift=2mm]{Offer} (ps6);
  \draw [->] (qs6) -- node[yshift=-2mm]{Notif} (q6);
  \draw [->] (p7) -- node[left,yshift=2mm]{Offer} (ps7);
  \draw [->] (qs7) -- node[right,yshift=-2mm]{Notif} (q7);
  \draw [->] (p8) -- node[left,yshift=2mm]{Offer} (ps8);
  \draw [->] (qs8) -- node[right,yshift=-2mm]{Notif} (q8);

  \path (rec3) +(8mm,10mm) coordinate(x1);
  \path (rec3) +(20mm,10mm) coordinate(x2);
  
  \draw [-to,thick,segment amplitude=.4mm,
         segment length=2mm]
         (x1) -- (x2)
         node [above=1mm,midway,text width=2cm,text centered]{decentralization};

\end{tikzpicture}
\end{adjustwidth}
\caption{\small Parallel distributed system with centralized scheduler}\label{fig:mul}
\end{figure}
\begin{example}
  Figure \ref{fig:mul} depicts the result of the transformation of the timed system of
  Figure \ref{fig:run} towards a distributed model with a centralized scheduler. 
  This transformation introduces parallelism between components, that is,
  the components involved in the selected interaction will execute in parallel.
  For example, if the scheduler chooses to execute the interaction
  $init_1$, both components $C$ and $T_1$ will execute this corresponding actions in parallel.
\end{example}

